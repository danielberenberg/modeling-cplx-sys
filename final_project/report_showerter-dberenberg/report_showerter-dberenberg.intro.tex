Our main interests in this research are the translation of complex voter opinions into the voting arena and the influence of different voting systems such as ranked-choice or approval voting on voter satisfaction of election outcomes.
We are defining ``complex voter opinions'' as the taking into account dimensionality of opinion space as being higher than 2D.
Most opinion models work off of the notion that opinions are two-dimensional and often binary.
We were interested, for the modeling of general elections of the idea that our opinions are not two-dimensional and are far from binary by having voter's opinions exist on a multidimensional vector, where each dimension symbolizes a distinct but potentially important topic that one can form an opinion on.

Additionally, we sought to compare different voting systems, utilized in different countries around the world, within this framework of multidimensional opinion space.
The voting systems were implemented were plurality (the system in most U.S. elections), ranked-choice (used in national elections in Australia), and approval voting (utilized by many academic and research associations and institutions).

While exploring these different systems, we also look at the idea of campaign strategy and opinion transparency of candidates by having candidates full opinions not always be exposed to the voting population.

\todo{write more about why this is interesting}\\
\todo{contributions of our research to the field / some literature review}\\


\cite{hoyer1974comparing}
