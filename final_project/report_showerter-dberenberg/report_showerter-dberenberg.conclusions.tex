%%%%%% Discussion and conclusions

While ... we think ....

In future work, we would like to further explore this model by informing our opinion vectors with data.
We think it could lead to even more interesting results if the opinions were drawn from different distributions for the different topic dimensions, possibly distributions representative of actual polling data on a variety of political topics.

For instance, the effect of the candidate opinion transparency would likely be much greater if all topics were not distributed the same in the opinion space, such that if one topic was heavily polarized and bi-modal and was masked by a candidate, it could lead to a much larger change in happiness outcome.

Much of the weakness of this study is in the purely stochastic set-up, where all seems to average out in the end.
But even though it was not heavily significant, the plurality voting system was the worst for overall happiness in every scenario run, and surprisingly approval voting, which is similarly simple came out ahead a couple times.
