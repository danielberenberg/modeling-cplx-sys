\renewcommand{\algorithmicrequire}{\textbf{Input:}}
A voting system $V$ is an algorithm that takes as input $N$ submitted ballot sheets and $k$ 
candidates from a population of individuals and returns a single winner 
of an election.
Two voting systems $U$ and $V$ may have unique input formats for their ballot sheets, (e.g, plurality and ranked-choice, see~\ref{sec:intro}
for a short description of each).
For a certain population $k$ individuals are chosen as candidates. 
Citizens are asked to cast their votes with respect the input requirements of $V$ and a voting system is employed across the population. \newline
\indent Evaluation of a voting system is performed by computing the distribution of happinesses/satisfaction/agreeability across a population  
with the winner of an election.
This metric is described in~\ref{sec:model}.
\newline
\indent The voting systems that will be tested are plurality, ranked-choice, and approval voting, to be described in the following sections.
\subsection{Plurality}
Each individual casts a single vote for the candidate with whom they believe their opinion is aligned most.
Votes are tallied and the candidate with the largest share of votes wins the election.
\begin{algorithm}[H]
\caption{Plurality Voting System Algorithm}\label{alg:plurality}
\begin{algorithmic}
    \Require Set of $k$ candidates $C=\{c_1, \ldots c_k\}$, ballot box $B$
    \Return $\mathrm{argmax}_{c\in C} \Call{tally}{c, B}$
    \Function{tally}{candidate, ballots}
    \State \Return $\sum_{b \in B}\mathbbm{1}_{b = \mathrm{candidate}}$ 
    \EndFunction
\end{algorithmic}
\end{algorithm}
The \texttt{plurality} algorithm runs by finding the candidate that maximizes the \texttt{tally} function, a counter for the votes
for that candidate.
\subsection{Ranked-Choice Voting}
Each individual ranks all candidates from most to least favorite.
Votes are tallied to compute the proportion of the population that voted for each candidate.
If the maximum proportion is less than majority ($\geq 51\%$), the bottom candidate is removed from the race and their votes are redistributed.
Redistribution takes place by observing the next highest ranked candidate in each of the rank sheets submitted by those who casted votes
to the removed candidate. This process goes on until one candidate has attained the majority share of votes. \newline
\indent In ranked-choice voting, the algorithm takes as input a ballot box $B$ of arrays $b$ of size $k$, where $b_i[0] = $ the ideal candidate for 
citizen $i$.
\begin{algorithm}[H]
\caption{Ranked-Choice Voting System Algorithm}\label{alg:ranked}
\begin{algorithmic}
    \Require Set of $k$ candidates $C=\{c_1, \ldots c_k\}$, ballot box $B$ of ranked choices.
    \Function{tally}{$C, B$}   
        \State votes $\leftarrow \{b[0] \mid b \in B\}$
        \If{$\max \{$\Call{voteShare}{$c$,votes}$| c \in C\} \geq 0.51$}
            \State \Return $\mathrm{argmax}_{c \in C}$ \Call{voteShare}{$c$,votes}
        \Else
            \State bottom $\leftarrow \mathrm{argmin}_{c \in C}$ \Call{voteShare}{$c$,votes}
            \State $C \leftarrow C \setminus \{\mathrm{bottom}\}$
            \For {$b \in B$}
                \If {$b =$ bottom}
                    \State $b \leftarrow b[1:]$ \Comment{Remove the first entry of $b$}
                \EndIf
            \EndFor
            \Return {\Call{tally}{$C, B$}}
        \EndIf
    \EndFunction
    \Function {voteShare}{candidate, votes}
        \State \Return $\frac{\sum_{b \in B}\mathbbm{1}_{b = \mathrm{candidate}}}{|\mathrm{votes}|}$
    \EndFunction
\end{algorithmic}
\end{algorithm}
The algorithm proceeds by recursively calling \texttt{tally} on its candidate and ballot sheet sets until a candidate has met
the desired popular vote threshold of $0.51$.
\subsection{Approval Voting}
In approval-based voting, an approval threshold is first chosen. Our work focuses on an approval threshold of $1$, the reason
this being that our metric for the disagreement (or dissimilarity) between two opinions is bounded in $[0,2]$ and a threshold of 1 implies that
no voter will agree with a candidate ``across the aisle''. Given the approval threshold, the citizen submits as an unordered set of candidates
that meet that threshold with respect to dissimilarity.
The winner of the election is selected by finding the candidate with the most votes. 
Hence, the algorithm for approval is almost exactly the same as for plurality with the main difference being that approval
requires a collapsing step in which all ballot sheets are submitted and aggregated into one long list of votes.
One characteristic of approval based voting is that vote percentages may not add up to 100\%. 
This does not change the way the election is held but should be taken into mind when computing vote related statistics. 

% FYI in the revtex environ doing figure* makes your fig span both columns
% and I think the auto placement is the top of the next page.
